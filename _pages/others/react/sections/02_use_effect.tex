You can think of \texttt{useEffect} as \texttt{componentDidMount}, \texttt{componentDidUpdate},
and \texttt{componentWillUnmount} combined. 

\subsection{Effect without cleanup}

Sometimes, we want to run some additional code \textbf{after} React component
has updated the DOM. 
In React the \texttt{render} method, shouldn't cause any side effect. We usually
want to perform effect \textbf{after} React has updated the DOM. 

This is why usually we put side effects in \texttt{componentDidMount} and 
\texttt{componentDidUpdate}. 

An example is:

\begin{lstlisting}
class Example extends React.Component {
  constructor(props) {
    super(props);
    this.state = {
      count: 0
    };
  }

  componentDidMount() {
    document.title = `You clicked ${this.state.count} times`;
  }
  componentDidUpdate() {
    document.title = `You clicked ${this.state.count} times`;
  }

  render() {
    return (
      <div>
        <p>You clicked {this.state.count} times</p>
        <button onClick={() => this.setState({ count: this.state.count + 1 })}>
          Click me
        </button>
      </div>
    );
  }
}
\end{lstlisting}

