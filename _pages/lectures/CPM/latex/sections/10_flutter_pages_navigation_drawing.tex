\subsection{Pages}

Page or screen is a Widget tree aka a Route. 

The App widget defines the home route and others, from Home, routes, and initialRoute 
properties (on the App constructor).

The home is the initial screen (widget tree) shown. 
Alternatively, a set of routes (pages) can be defined and an initialRoute specified.

\subsection{Navigation}
The Navigator widget allows the replacement of a page by another using a stack discipline.
It is possible to create a set of navigation routes previously in the app, or build each one
when we want to navigate to it. Initially the App has already a Navigator. We can use it or
create a new independent one with new routes. 

The Navigator defines a set of static methods that manipulate the stack of pages (routes):

\begin{itemize}
    \item \texttt{push(context, Route)} and \texttt{pop(context)}
    \item \texttt{pushName()}, \texttt{removeRoute()}, \texttt{replace()},...
\end{itemize}

\subsubsection{Prebuild route table}
It's possible to prebuild a route table in the App widget and put it in the routes property: 
\begin{lstlisting}
void main() { 
    runApp(MaterialApp( 
        home: MyAppHomePage(), // becomes the route named '/' 
        routes: { 
        '/a': (BuildContext context) => MyPage1(title: 'page A'), 
        '/b': (BuildContext context) => MyPage2(title: 'page B'),   
        '/c': (BuildContext context) => MyPage3(title: 'page C'), }
    ));
    } 
\end{lstlisting}

\subsubsection{Create pages dynamically}
New screens or pages (widgets trees) can be built from a PageRoute (derived from)
e.g., the MaterialPageRoute, requires a WidgetBuilder (a function) that builds the tree
from context. 

\begin{lstlisting}
Navigator.push(context, 
MaterialPageRoute(builder: (context) => MyPage1(title: "page A")));
\end{lstlisting}

\subsubsection{Navigation forward and backward}

\begin{lstlisting}[title=Navigate forward]
// within a widget
... 
Navigator.pushNamed(context, '/b');
\end{lstlisting}

\begin{lstlisting}[title=Navigate backward]
// within the navigated page
...
Navigator.pop(context);
\end{lstlisting}

\subsubsection{Passing data to a route}

\begin{lstlisting}[title=Class to be passed]
class ScreenArguments {
    final String title;
    final String message;
    ScreenArguments(this.title, this.message);
} 
\end{lstlisting}

\begin{lstlisting}[title=Class to create page]
class HomePage extends StatelessWidget {
    @override
    Widget build(BuildContext context) {
        return ...
    ...
    onPressed: () {
        Navigator.pushNamed(context, '/page2',
        arguments: ScreenArguments('Title', 'Message'));
    }
    ...
    }
}
\end{lstlisting}

\begin{lstlisting}[title=Page]
// App root widget
...
return MaterialApp(
    routes: { '/page2': (context) => SecondPage() },
    home: HomePage() // route name: '/'
)
...
class SecondPage extends StatelessWidget {
    @override
    Widget build(BuildContext context) {
        final ScreenArguments args = ModalRoute.of(context).settings.arguments;
        return ...
            ... args.title ...
            ... args.message ...
    }
}
\end{lstlisting}

\subsubsection{Return a value using pop}
When we do a \texttt{Navigator.pop(context)}, \texttt{pop()} accepts a second parameter that is a
result. It can be of any type.

This result is returned by the \texttt{Navigator.push(...)} that created the popped page.
The result comes in a \texttt{Future}, resolved only when the page is popped.

\begin{lstlisting}[title=Example]
// caller 
final result = await Navigator.push(context, MaterialPageRoute(builder: (context) => SomePage())); 

// New page - SomePage
Navigator.pop(context, 'Some message');
\end{lstlisting}


\subsection{External HTTP Calls}
Flutter does not have a direct API for web requests.
But there are several packages in the package repository (pub.dev).

One of the simplest is the http package (https://pub.dev/packages/http)
For using external packages, it is necessary to import them in code, and to
declare them in the project pubspec.yaml file. Finally, it must be downloaded.

\begin{tcolorbox}
The declaration and installation can be done from the command line, or from the
used IDE:

\texttt{\$ flutter pub add http}
\end{tcolorbox}

For example, to define a function to perform a REST GET request:

\begin{lstlisting}
    import 'package:http/http.dart';
    ...
    Future<String> getResponse() async {
    final response = await http.get(Uri.http('data.fixer.io', '/api/latest',
        { 'access_key': '<your API key>',
            'base': 'EUR',
            'symbols': 'USD,GBP' 
        }
    ));
    if (response.statusCode == 200)
        return response.body;
    else
        throw Exception('HTTP failed');
    }
\end{lstlisting}
