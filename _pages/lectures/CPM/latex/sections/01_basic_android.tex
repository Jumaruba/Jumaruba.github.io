The android framework for building applications was developed in C and wrapped in java. 
The java virtual machine was also written in C and optimized for small devices and cores. 
The framework itself was based in reuse and extension patterns.   

The \textbf{application framework} is a \textbf{high level library in java} suitable to the creation
of user android applications. In other words, it's the basic structure underlying the system (Android API).  
The framework also creates the higher-level Android management services (Java).  

\begin{itemize}
    \item Activity manager
    \item Content manager 
    \item Resource manager 
    \item Location manager 
    \item Notification manager
\end{itemize}

\subsection{Application components}
The android applications can contain serveral independent components:  
\begin{itemize}
    \item Activities 
    \item Services 
    \item Broadcast Receivers 
    \item Content providers 
\end{itemize}

\subsubsection{Activities}
\begin{itemize}
    \item Contain a portion of the user interface
    \item Execute a well-defined task inside the application 
    \item Are subclasses of the android.app.Activity class
    \item Are usually composed by a hierarchy of Views
\end{itemize}

\subsubsection{Services}
\begin{itemize}
    \item Don't have user interface, like the activities
    \item Can execute in background for indeterminate period
    \item It's possible to estabilish a connection with the service and communicate through a
    well-defined programming interface
\end{itemize}

\subsubsection{Broadcast}
It handles notifications and the applications react to then. 
\begin{itemize}
    \item Can receive and react to notification received from the system or other applications 
    \item Applications can initiate a notification 'broadcast' 
\end{itemize}

\subsubsection{Content providers} 
Basically allow the comunication of data between applications.
\begin{itemize}
    \item Make available to other applicationo a data collection maintaned by this application
    \item Define an interface to access, add and update the supported data types
\end{itemize}

\subsection{Component activation}

\subsubsection{Activities and Services}
\hl{Activities and services are activated through an \textbf{Intent}.}

The \textbf{intent} identify the component or specify actions for activities and services.  

A explicit intent contains the:
\begin{itemize}
    \item Class reference of the desitnation
    \item Specification of an action, category, data (in the form of url) and possibly extra information.
\end{itemize}

Once the specifications are given, the android will try to find a matching component capable of executing ht eaction in the data (or data type) specified.
When activities and services are declared in the \textbf{manifest} they can specify the \textit{intent-filters}, describing their accepting intents.

An example is calling the activity using an intent:
\begin{lstlisting}
val otherActivity: Intent = Intent(thisActivity, OtherActivity::class.java);
thisActivity.startActivity(otherActivity);
\end{lstlisting}

\subsubsection{Broadcast receivers}
The intents for activating broadcast receivers indentify a \textbf{message} to be delivered to matching receivers.
The message is specificied using: action, category, data and extra info.

\begin{lstlisting}
sendBroadcast(Intent);      // Sends the intent.
onReceive(Context, Intent); // If a broadcast receiver was installed and matches the intent, this method will be called.
\end{lstlisting}

\subsubsection{Content providers}
When declared in the manifest must have an 'authority'. The 'authority' must recognize a name for 
its data collection. 

The content providers supports the CRUD operations on the specified collection. 

They are usually activated by the method \textbf{getContentResolver} and the \textbf{ContentResolver} can
insert, update, delete and query an information.

\begin{lstlisting}
ContentResolver resolver = getContentResolver();
Uri uri = Uri.parse("content://<authority>/<data-collection>[/<item>]"); 
resolver.query(Uri uri, String[] projection, Bundle queryArgs, CancellationSignal cancellationSignal);
resolver.delete(Uri uri, Bundle extras); 
resolver.insert(Uri uri, ContentValues values);
resolver.update(Uri uri, ContentValues values, Bundle extras);
\end{lstlisting}

\subsection{Intent (start an activity example)}

All the intents have a name (action) and can gave nire data associated (e.g uri, category, extra info). 

\hl{The intents can also be \textbf{explicit} with a class reference (inside an app).}

There're many predefined intents. 

This first example matches the android activity in the Dialer application, that allows the 
user to make a phone call, declared in an intent-filter, that can handle this action.

\begin{lstlisting}
val intent = Intent(Intent.ACTION_DIAL);
startActivity(intent);
\end{lstlisting}

This second example, matches an activity that makes a phone call, but for a given number.
\begin{lstlisting}
val intent = Intent(Intent.ACTION_CALL);
Uri.parse("tel:555-555-555").also{intent.date = it};
startActivity(intent); 
\end{lstlisting}

Creating an intent to start an activity externally the app: 
\begin{lstlisting}
Intent LaunchIntent = getActivity().getPackageManager().getLaunchIntentForPackage(CALC_PACKAGE_NAME);
startActivity(LaunchIntent);
\end{lstlisting}


\subsubsection{Intents data}
\hl{Intents \textbf{can transport data between components}}. In other words, if the application is currently running in an activity,
we can initiate another activity and parse the actual information to it by using intents. 

The \textbf{data} property can be used for any kind of a Uri. The calling and new app components uses the Intent \textbf{data} property
of the Uri type, by calling \textbf{setData()} and \textbf{getData()} methods to set it. 

The Extra internal field is used for arbitrary data types. 

It is a \textbf{Bundle} : set of (name, value) pairs organized in a hash table.
\begin{itemize}
    \item The value can be a String, simple type, or an array.
    \item Can also be any Serializabe or Parcelable (more efficient) object
    \item The values are inserted with some \textbf{putExtra(String name, ... value)} method and retrieved
    with \textbf{get...Extra(String name)}.
\end{itemize}

\begin{lstlisting}
override fun onCreate(savedInstanceState: Bundle?){
    super.onCreate(savedInstanceState);
    ...
    /* Activities have an intent property */ 

    val fName = intent.getStringExtra("firstName");
    val lName = intent.getStringExtra("lastName");
}
\end{lstlisting}

\begin{tcolorbox}[colback=red!5,colframe=red!75!black, title=Disadvantages of using Bundle]
When used to pass information between Android components the bundle is serialized 
into a binder transaction. The total size for all binder transactions in a process is 1MB. 
If you exceed this limit you will receive this fatal error: \texttt{Failed binder transaction}. 

It's recommend that you keep the data in these bundles as small as possible because 
it's a shared buffer, anything more than a few kilobytes should be written to disk.
\end{tcolorbox}

\subsubsection{Activity retuning data}
\hl{Specially invoked activities \textbf{can return data}.}
This can be done by calling a new activity with \textbf{startActivityForResult(...)}.

Besides the intent, it has a requestCode(Int) as a parameter. The new activity should create 
a \textbf{result Intent}, fill it with the result data, and call \textbf{setResult}, passing this 
intent before finishing. 

Called activity sending value: 
\begin{lstlisting}
...
val resultIntent = Intent();
resultIntent.putExtra("some_key", "String data");
setResult(Activity.RESULT_OK, rsultIntent);
finish();
\end{lstlisting}

Original activity receiving value:
\begin{lstlisting}
override fun onActivityResult(requestCode, resultCode,: Int, data: Intent){
    super.onActivityResult(requestCode, resultCode, data);
    when(requestCode){
        MY_CHILD_ACTIVITY ->
            if(resultCode == Activity.RESULT_OK){
                val resultValue = data.getStringExtra("some_key"); 
                ...
            }
    }
}
\end{lstlisting}


\subsection{Colletion of elements}
In android we might want to display a collection/list of elements to the user and this 
list might suffer some updates.  
\hl{For this, we must use the \texttt{ArrayAdapter} class.}
If we want to personalize the adapter, we must override this class. 

\begin{lstlisting}[title=ArrayAdapter Class Implementation]
class MainActivity : AppCompatActivity() {
    private val adapter by lazy { RestaurantAdapter() }
    
    override fun onCreate(savedInstanceState: Bundle?) {
        super.onCreate(savedInstanceState)
        supportActionBar?.setIcon(R.drawable.rest_icon)
        supportActionBar?.setDisplayShowHomeEnabled(true)
        setContentView(R.layout.activity_main)
        app.adapter = RestaurantAdapter(this, app.rests)
    
        val list = findViewById<ListView>(R.id.listview)
        list.adapter = app.adapter
    } 

    
    inner class RestaurantAdapter : ArrayAdapter<Restaurant>(this@MainActivity, R.layout.list_row, rests) {
        override fun getView(position: Int, convertView: View?, parent: ViewGroup): View {
            val row = convertView ?: layoutInflater.inflate(R.layout.list_row, parent, false)
            val r = rests[position]
            row.findViewById<TextView>(R.id.title).text = r.name
            row.findViewById<TextView>(R.id.address).text = r.address
            val symbol = row.findViewById<ImageView>(R.id.symbol)
            when (r.type) {
                "sit" -> symbol.setImageResource(R.drawable.ball_red)
                "take" -> symbol.setImageResource(R.drawable.ball_yellow)
                "delivery" -> symbol.setImageResource(R.drawable.ball_green)
            }
            return row
        }
    }
}
\end{lstlisting}
