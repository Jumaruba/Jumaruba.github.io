[slide 22]  Having a type inheritance is allowed.  

\subsection{Domain definition}
A type can be seen as a domain. 

A user can define a type as follow:
\begin{lstlisting}[language=sql]
    create type address\_t as object (
        street varchar2(30)
    );
\end{lstlisting} 

The \texttt{as object} declares that the table is actually an object. 
\textbf{Types does not take spaces}. They are just definitions. 


\subsection{Objects in table columns}
Types can be used a domaisn in table columns
\begin{lstlisting}[language=sql]
    create table members(
        ID number(10) primary key, 
        address address_t
    )
\end{lstlisting} 

As it's possible to see in the code above, the \texttt{members} talbe uses the type \texttt{address\_t}. The first column in the \texttt{members} table is a scalar value, but the second one is filled with objects. The objects are stored in the table row. 

\subsection{Object creation} 

Objects uses constructors to be created. 
\begin{lstlisting}[language=sql]
    insert into members(ID, address) values (3658, address\_t('Rua legal'));
\end{lstlisting} 

If we perform a query on a table that contains a column the result will be less detailed.
\begin{lstlisting}[language=sql]
select * from members; 
\end{lstlisting} 

But if we apply some recursion, we have more details: 

\begin{lstlisting}[language=sql]
    select id, m.address.street from members m where id=3658; 
\end{lstlisting} 


\subsection{object tables} 

In this case each row of the table is an object. 

\begin{lstlisting}[language=sql]
    create type department_t as object(
        acronym varchar2(5), 
        designation varchar2(20),
        director number(10)
    );
\end{lstlisting} 

We can use \texttt{departments} as a normal table, but we may force the object view:

\begin{lstlisting}[language=sql]
    select value(d) from departments d where director=3658;
\end{lstlisting} 

Usign value(d) we gonna gave different values. 

\subsubsection{References} 

A big difference from teh column objects is that in object-row we have addresses. To get this pointer, we use the \texttt{ref}. 

\begin{lstlisting}[language=sql]
    select ref(d) from departments d; 
\end{lstlisting}  

The interface will autoamtically the converts the address to the object. But in programming level, the difference can be seen. 

We can say that in a table we gonna add pointers to a table with a specific type:
\begin{lstlisting}[language=sql]
    alter table members add(department ref department_t)
\end{lstlisting} 

But how can we use the pointers? There are some queries that take advantages on this. 

\begin{lstlisting}[language=sql]
    update members set department = (select ref(d) from departments d where sigla='CLU' where id=3658)
\end{lstlisting}  

The pointers are defined to a type, not a table. The foreign keys are references a table, but with pointers it's possible to point to several tables.  

[Check slide 32].  

\subsection{Dereferencing an object} 
[slide 33] we are able to get a column from another table without performing a join. In the slides examples we're getting \texttt{m.department.director} column, which is in another table. 

